\chapter{從細菌到宇宙}
我道:「梅,你可信我知道全部?」

梅開道:「好,我相信。」

我又道:「梅,那你可信我不知道全部?」

梅開有些不耐煩:「我不明白你在說甚麼,甚麼知道不知道的!我希望你能表述清楚一些。」

我道:「或許你不這麼認爲,事實上,世界就是這麼處處關聯的,可以輕易地從四聲道立體聲音響想象到故歐洲歷史。這很正常,你有任何不滿可以向我提出來,我不介意,希望你也不要介意。」
%(出處:此處的古歐洲歷史到四聲道立體聲音響關係到衛斯理尋夢中的典故)王居風!!

梅開小聲說了些甚麼,我沒有聽清楚,但多半是對我的不滿。

我繼續道:「細菌於宇宙的關係,可以聯繫到我與你的關係。」

他一拍額頭,像是恍然大悟一樣,隨後又顯出失落的表情,我不知道那是甚麼意思。
%(出處:第一版中這裏梅開是在表示失望,他又要忍受亞首的毫無邊際的囈語了。)

梅開道:「爲何你不停止廢話,直接講到點上呢?」

我道:「這不是廢話,梅,這不是廢話。所謂哲學,大抵就是由這些組成的。」
%(出處:這是One Flew Over The Cuckoo's Nest裏Taber與Harding的對話。)

我看出梅開的不滿,道:「你若不喜歡聽下去,何必在此受這樣的『煎熬』,你大可直接離去,我不介意。」

梅開露出一絲很不好意思的表情:「沒關係,你可以繼續說,是我太沒有耐心了。」

然後梅開打開身旁的櫃子,取出兩瓶酒來,遞給我一瓶,我咬開瓶蓋,喝了一口,又抹了抹嘴。我很有必要整理一下我要講的東西,它們實在太亂了。

然後我道:「你也能懂的,只要人類還在進步。」

緊接着我又想到一個問題,便道:「但恐怕人類不會再進步了,人類已經到終點了麼?爲何拒絕進步,是卑劣的人性,還是旁的原因?」

梅開沒有動作,我則繼續講着我想到的東西:「一定是!人的劣根性遲早讓他們走向終點!」

梅開看似有些不滿:「莫忘了,你我也是人!」

這正讓我抓住些東西,我道:「我知道,有一個故事,不知道你是否聽過。某人談起自己時,若道:『我是一個勤勞的人』,那麼他一定是一個自大的傢伙,而如果他轉而道:『我們的民族是勤勞的民族』,那就截然是不同的結果了。喜歡這類說話方法的人,往往更加自大,而且決計忌諱別人談及他的缺點或者談及他沒有的缺點,就會非常暴躁,這類人完全沒有道理可講。正因爲這種人太多了,所以在中國罵人的話裏纔會總是和人家母親扯上關係。梅,我真不希望你也是這樣的人。」
% 這句話的出處是[眼睛]中G作家對「辣塊媽媽」的解釋
% 不,一點也不晦澀,大到種族,小到親戚,各種關係產生的榮耀或恥辱,都是一種反獨立現象,這在後面講及思想獨立的時候會詳細描述這種心理

梅開想了想:「你說的是,我想我已經不是這樣的人了。你可以繼續你的講述了。」
\\


我又喝了一口酒,思緒也活躍了起來,當我想及這個問題的原因時,我又激動了起來。

我大聲道:「人之劣根性,是因爲甚麼?他又如何知道?源自他存在!所有存在的,都有劣性,即使是一個原子,也是不應該的!最終是無!甚麼都沒有!」\index{劣根性}

我接着往下想,便想到了另一個問題,道:「無,甚麼是無?無充斥着一切?哈哈,那它的密度是多少。沒人知道,我不知道是不是零,我感覺應該是零。」\index{無}

梅開道:「首先,『無』不是物體,只是一種概念,沒有密度這一說。就如,你能說人的思想有多大密度麼?這點你應該比我還要清楚。」

我想着:「不,應該有的,我感到它有,而難以講述出來。」

梅開道:「你說的話很荒謬,但是我像我們不該討論這個問題了,『無』的密度究竟如何,實在是一點關係都沒有的。」

我想了一會,不知道該如何講述下去,於是我放棄了這個話題。
\\


我轉而道:「我要講述的是一套相對完整的思想,這些東西完全經由我的大腦思考出來,然而它幾乎是我的全部,你需要理解我的一些基本概念。」

梅開望着我的眼睛,道:「好。」
\\


我道:「首先,我希望得知你的想法,你是否認爲這個世界中擁有隨機的事件?」\index{隨機事件}

梅開幾乎沒有想就答:「自然有,譬如我們的一些思想,在一些事情中的決定,最顯而易見的,譬如布朗運動,這些都是隨機的。」

我道:「對於這類事件,我的看法不同,譬如我們思想中的決定,我們自然是在一件事情中已經有了權衡,纔會做出相應的決定的。又譬如若你讓我順口說出幾個隨機數,我做得到,然而我想你是研究過心理的,人在這種情況下對每個數字自然也會有不同的權衡,因而他們會根據這些權衡來說出他們需要的隨機數,往往這個權衡是在潛意識中做好的,所以當事人並不知道。」

梅開點點頭,道:「確實是這麼個情況,但是布朗運動一定是隨機的了。」\index{布朗運動}

我道:「不是這樣的,分子運動受到種種因素影響,譬如碰撞,譬如相互間的吸引,這些量是可測的,或者是可以在理論上分析的,在這些可測的因素下的運動,並不算真正的隨機,它只是在宏觀看來是一種不規則運動,而事實上,它受到的種種牽制纔是它看似隨機的原因。」

梅開道:「那世界上難道沒有甚麼事真正隨機的了?」

我想到一個證明方法,道:「假設,我只是假設存在時光機器,某時間,真正的隨機事件~A~發生了,若過後,我們乘坐時光機回到那個時間再對之進行觀察,隨機事件~A~是否還會發生呢?」

梅開道:「自然不一定。」

我道:「確實是這樣,因爲它是絕對隨機的,然而如果我加一個條件,假設我們通過時光機回到那時,可以完全不對當時的世界照成任何影響,我們只是觀察,甚至我們的觀察也完全對之沒有影響,這只是假設,這是很難成立的。」

梅開想了想道:「我想是的,隨機事件~A~既然是隨機的,那麼它肯定不會有確定的結果。」

我道:「不然,我的條件是完全沒有影響,我們在一條時間綫上,如果~A~在某一個時刻發生了,而在相同的時刻同時沒有發生,它豈不是在這個時刻處於一種發生-未發生疊加的狀態?」
% 薛定諤的貓

梅開道:「可以這麼說,但那樣似乎不得不創建另一個平行宇宙來容納你說的這種疊加情況。」\index{時間岔道}

我很興奮:「對!但是從我們看來,我們是一條時間綫上的,從頭到尾,一切的一切,依然是固定的,而且不容改變的,如果改變了,那便是另一條時間綫上的了。我們無論如何向上追溯,依然是得到一條時間綫,因爲我們無法突破時間的岔道,來到平行宇宙,那需要真正的時光機才行。」

梅開沉思了好一會,陡然道:「你說的是真的!我們的世界並沒有真正的隨機,一切都是固定的。天啊!這太恐怖了!」

我道:「你懂了這一點,就好多了,那麼這麼看來,你可以知道,我們的一切都是已經註定好的,一切都是完全設置好的,你有甚麼概念。」

梅開大聲道:「宿命!宿命!」

我道:「這種平行宇宙的創建,需要一個條件,那就是真正的隨機事件~A~,而據我瞭解,似乎還沒有甚麼事件是絕對隨機的,或者量子尺度上會有,或者在宇宙之外會有,但就目前而言,我不知道有這樣的東西可能存在。」
% 量子尺度上確實有,但是那不在突變的範疇之內,事實上,突變理解的量子就是同時多個平行宇宙在極小的範圍中「疊加」狀態。

梅開想到一個問題,他問我:「那麼,理論上是否存在預言家\index{預言家}呢?」

我道:「絕對存在,我們存在於這樣一個有規律的宇宙,一切現象都有規律可循,自然可以預言,就像我看到一個人從十層樓跳下來,可以很自然而然地預言他會被摔死,這是很正常的。」
% 那個人會不會是我?我想不會,我會找一種比較快速而且舒服的方式自殺,譬如觸電等等。

梅開解釋道:「我是指另一種預言家,人類一直在探究的,八卦\index{八卦},占星學\index{占星學}等等這類。」

我道:「那有甚麼區別呢,我可以告訴你一種預言宇宙之間一切事件的方法,但有一個前提,你必須懂得宇宙間所有規律。」

梅開問:「那有甚麼用呢?」

我哈哈道:「那麼你就可以計算出每一個基本粒子的軌蹟,在你要預言的那個時刻它在哪裏,把那一系列基本粒子組合起來,你就可以精確預言那一件事情了。不過前提是你必須知道宇宙間所有基本粒子的信息,你需要得知他們的相互影響。」\index{準確預言法}
% 我真希望我們HOST計算機爲了提高算法效率,而採取不完全計算攝動的方法,那麼進行預言可能會容易得多。看來的確實可能是這樣的,量子尺度上已經顯現出了這種不精確。

% Checked here. Thu Dec 1 6:47 AM (2011) -Shou Ya.

梅開聽出我的調侃,又道:「這說法太荒謬了,我是在認真的問,事實上,我雖然不太相信那些玄學,但總是還抱有一點點將信將疑的態度。」

我道:「預言其實有捷徑可走,譬如我要計算星球運動,我只需要設法計算出對它攝動較大的星體的質量,它的質量,基本上就可以預言了,這樣做的結果是犧牲了很大的精確度,就如我無法用這種方式計算出它某一個時刻的某個粒子在甚麼位置一樣。但是無疑我們不需要那麼大的精確度,所以我們可以有所這些精度犧牲,我們可以較快的計算出我們滿意的數字。而星相法和八卦,我不太認可,我不認爲這是一種很好的捷徑,我覺得,預測人和預測星體一樣,需要找出對它攝動較大的對象,譬如預言一個人的一生作爲,可以考察其朋友,其家庭,其工作環境來綜合考慮,所以我覺得你們這類心理學家要比那些八卦先生準確多了。」

梅開追問:「可是他們可以預言譬如明天的一場車禍甚麼的,這些看似很隨機巧合的事件,我卻不認爲考察他身邊有些甚麼樣的朋友就可以預言到這些的。」

我道:「我並沒有否認這類玄學者,他們或者會有道理,但是我從不認爲他們能做到。這類不相干因數很多的事件,難道你認爲會和星星的位置,和他們的生辰年月有甚麼關係麼?預測這類意外,我認爲幾乎是毫不可能,至少我覺得那類玄乎之至的理論完全不符合我的邏輯。」

梅開想了想,也接受了我的想法,他指着我的鼻子道:「你倒真是個典型的唯物主義者\index{唯物主義者}。」

我道:「你可以這麼說,我極度唯物,而我同時是極度的唯心主義者\index{唯心主義者},而且它們完全不矛盾。」
%(出處:我在政治手冊哲學章上的標語)

梅開想問我甚麼,但被我打斷:「我會一一告訴你的,但是如果現在告訴你,你或者完全無法接受。」

