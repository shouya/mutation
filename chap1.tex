\chapter{精神病人}
我是亞首\index{亞首},現代人。

我的職業是天體物理學研究員,在一家天文觀測站工作,絕對是一個平凡的人。

我和你們一樣,迷信科學,相信科學可以解釋並且解決一切問題,自然,到目前爲止,我依然如此認爲。我雖然有的一些奇特的思想,但顯然與此不悖。
\\


不知什麼時候,我被作爲精神病人\index{精神病人}送達市精神病院的,我清楚我的處境,但我不明白爲甚麼。

我思考可能是因爲我常常間歇性失去一些時間,事實上我都不能確定我是否確實失去過那些時間,我依然是我,我知道我做的一切,然而我的思想不是。

在那些時候,我總是會有些古怪的想法,我記不清楚那是甚麼時候開始的。當我有那些想法的時候,它就像病毒一樣侵佔的我的大腦,我知道它的內容,但完全被它所控制,我明白它只是一種思想而已,它是無害的,看似如此,也確然如此。

我並不受控制,這思想雖然古怪,但很明顯是由我自己思考得出的,並不是被侵入的。這種思想一開始只有一點點痕蹟,後來逐漸有了完整的系統,甚至如今,我都不知道我是否分裂爲了兩個人,更不用說其中哪個纔是真的我。

很難使你瞭解這種狀況,因爲你並不是這樣的。

我不得不再次聲明一下,那思想\emph{絕對完全}是我的東西,就像耳朵一樣,絕不會有人抱怨它佔了腦袋的兩側位置的。
\\


我來到精神病院,是被帶來的,我被斷定爲明顯的精神分裂,而且是幾乎毫無希望的那種。

我清楚我並沒有病,之所以如此,可能是因爲我的想法太過古怪,甚至對生活都產生了一定的影響。在看來,這種狀況就是精神分裂。

我來到醫院之後,送我來的親人也都離去了,看得出來,他們的臉上呈現的不是悲傷,而是茫然。

我可以說是不完整的人了,在我的生活變得極度單調之後,我失去了很多東西,譬如方向感,敏捷的反應能力,我甚至記憶模糊。我感覺得到大腦在一步步退化,唯一還保留的,是邏輯思考能力,這一方面倒是逐漸強了起來。
\\


雖然我強烈表示我沒有病,但依然被帶到了病房,險些還要被用鎮定劑,因爲我可能看上去很激動,甚至暴躁,但這沒有什麼,不是麼?

我清楚我的處境,我只是想休息一下,並沒有料到結果竟然會這樣。

我很難證明我是無病的,於是我儘量表現正常,我按時吃飯,睡覺,無聊的時候,便在窗前踱步。

沒有人注意我,沒有人發現我是正常的,他們根本不在意,也根本不認爲我是有可能正常的。

我就在這樣的情況下在醫院住了一個多月。
\\


在這樣空曠的環境,我常常無法控制天馬行空的想象,我思考着那具體思想的各個方面,直到有一天,我突然想起一個人來。

我認爲他可能能夠幫我,他能理解這些,這個人的名字是「寧荒」\index{寧荒}。

寧荒是我的老朋友,我是如此認爲的。但事實上,我們只見過一面。
\\

在我還在讀中學的時候,有一次學校通知我們,附近社區請來一位大學教授,講授科普知識,內容是關於愛因斯坦的相對論的。至今我還清楚的記得,那是在週六的下午。

我在那個年紀,好奇心很強,加上我一向是比較喜歡這類科學的,自然是第一時間去報名了。因爲要聽這樣的講習,需要花來回的兩元車費,所以大部分同學都是放棄的,他們本來就沒有多大興趣,加上他們絕不會願意以那些東西佔用他們的寶貴的休息時間,他們更寧願省下兩元錢來買汽水喝。

我興致勃勃,很早就到達了,那場地是一間大的禮堂,座位很多,我找了個靠前位置坐下,等候開始。

這時候寧荒坐到了我的旁邊。我們是同齡人,從他的目光裏可以看得出他是一個熱愛求知的人。我們自然而然聊的很合,我們交流的內容從科學,到歷史,到哲學,涵蓋我們知識的各個方面。他很有知識,對一些問題也有自己的見解。在那個時候,找到一個這樣的朋友很不容易。我們非常開闊,那是我第一次那麼敞開心懷,那種交流的順暢,那種輕鬆,帶來的感覺,至今難忘,我相信他一定也有相同的感覺。
% 我和李輝或梅自薦的交流就有那種暢快感

我們幾乎完全沒有聽講習,因爲我們自講習開始到結束一直在交談。那位教授陶醉在自己的演講中,我們則沉浸在那種暢快的交流之中。

這就是我於寧荒認識的經過,但我們在忘記了一件事,我們本來就不會在意的,但後來想起來感到後悔莫及——我們忘記互相留下聯繫方式了。因而,我們事後再沒有機會聯繫。

這就是我和寧荒認識的全部經過。
\\


我想起他來的時候,情緒異常激動,我感到難以控制自己的想法,我不住想着希望再次見到他,我在想得入神的時候,我竟然大喊了起來,隨之,我有種近乎發狂一樣的激動。

看護者試圖穩定我的情緒,但看來拿不太可能,他不得不喊來我的主治醫生。而在主治醫生來到時,我的心境已經平緩了。我已經令自己冷靜下來了。

我向我的主治醫生道:「醫生,能否讓寧荒來?」

我看到他稍微猶豫了一下,點點頭:「完全可以,明天,明天怎麼樣,我一定讓他來看你。」

聽到他這麼說,我自然很高興,道:「那就好。」
\\


在醫生離開之後,我繼續想象着。隨即我就發現了一個問題,我從未向任何人提過寧荒這個人,爲何這位醫生會認識他?

這一個問題很值得懷疑,我的初步結論是寧荒是醫生的一個熟人。

但這個問題很快被我否決了,若醫生認識他的話,聽到我的話應該會驚訝我也認識他。他應該問「是那個寧荒麼?」或者「你也認識他?」之類的話,然而,他很爽快地回答了,這就證明這個猜想是不大可能的。

我進一步推測,唯一的一個可能就是:我的主治醫生在欺騙我,他根本不認識寧荒,也根本不可能讓寧荒來,他在敷衍,可能只爲了安定我,恐怕我被拒絕後情緒再度緊張。

而明天,一定會有甚麼事情發生的,這是他指定的日期,我倒要看看在那一日他無法帶來寧荒,會如何向我交待。
\\


果不其然,當天下午,我就被看護者帶去了一個地方。

那是醫院的一間類似小客廳的房間,有着一套軟沙發,旁邊的玻璃櫃子上擺設着一些作爲裝飾的酒,四周的牆上貼着一種不明顯但看上去很舒服的藍色壁紙,給人以非常平靜的感覺。

我等了不大一會,主治醫生也來了,他吩咐着我的看護者出去,便在我身邊的沙發上坐下了。
\\


他問候道:「哈!亞首,你還好麼!很高興見在這裏見到你。」

聽他的話,好像我和他是第一次見面一樣,而事實上絕非如此,我稍微有點猶豫。

我想起了上午的設想,這些人都是善於欺騙的,那麼他一定是在試探我的記憶能力。
% 按:對於亞首來說,任何人都是善於欺騙的,哈哈!或者應該說「任何物」更恰當一點。

而我,則把想法說了出來:「我不是在早晨剛見過你麼?真奇怪。難道是……」

他的表情看上去很滿意,笑道:「難道甚麼?」

我道:「難道你把我當成精神病人,在試探我?」

這話倒是令他驚愕了一下,顯得不知所措。

他只是呆呆道:「是……」

我便自然地笑了笑:「沒有關係的,醫生,你有任何問題,儘管向我發問。請不要把我當作一個精神病人了。」

他想了想,便點了一下頭。道:「嗯……我叫梅開,你的主治醫生,我不明白……你,你究竟是……」

我道:「我自然不必介紹了,我相信你在檔案上可以看到我的詳細信息的。」

梅開道:「是,是,只是你……你到底是怎麼回事?」

我道:「我沒有甚麼啊!你是在問我的表現不是一個精神病人所應該有的麼?」

梅開已經鎮定下來了,問:「對。你很奇特,你雖然作爲一個精神病人,如何能像正常人一樣,甚至比正常人還要敏捷,準確?」

我感到這個問題從他口中提出非常可笑,忍不住哈哈笑道:「你作爲一個精神病人的醫生,竟然如此直接詢問病人因何得病?」

梅開道:「自然不是問他爲何得病,而是因爲這個病人在不可能的情況下似乎康復了。」

我回想起來自己的過去,確然不覺得我有任何不同於常人之處,我便道:「當然不是自己康復了,或者……是誤診?」

梅開道:「我看過你的檔案,那顯示出的你確然擁有嚴重的精神分裂症,而……算了,事實就在面前,管他甚麼檔案,自然是那庸醫誤診的緣故。」

我聽到他這麼說,自然很高興,我道:「我自由了麼?我可以離開那個牢房也似地病房了麼?」

梅開道:「自然可以,但是你依然不能出院,你還需要繼續觀察。」

我有些不滿:「那有甚麼區別!你已經看到我現在這個樣子了,我正常的很,不是麼?」

梅開道:「是,可是……」

我打斷他的話,叱道:「難道你認爲我是個間歇性病人,會時而變成一個白癡麼?」

梅開無奈地攤了攤手:「不瞞你說,確然如此,我沒有辦法不這麼認爲。」
\\


我道:「我理解你,但這之所以如此完全是因爲你不理解我。」

梅開倒很直接:「當然,如果我能理解你,還需要藉談話的名義來與『開導』你麼?」

我垂下頭,梅開的坦誠,倒令我很喜歡他,我在考慮是否願意和他來共享我的想法,他是否有可能接受得了。

我望了他一眼,他很年輕,不會比我大兩歲以上,而且從他的目光中可以看出他的性格,誠懇,開朗,樂觀,當然更重要的是,他富有想象力。

於是我決定,如果他接受,我便讓他知道一切。

我擡起頭來,盯着他:「你可相信有人能改變自己?」\index{改變自己}

他沒有考慮多久就道:「你的意思是你的情況是你自己把自己變成這樣的?」

梅開竟然能在如此短時間內猜到我的意思,而且如此一針見血的指出來,實在令我對他的好感又增了幾分。

我道:「你終於理解了!你見過和我一樣的人?」

他還沒有回答。我已經迫不及待了:「有麼?是誰,我可以知道麼?」

梅開道:「沒有,你是我遇到過最特殊的病人,不,似乎不應該再稱你爲病人了。」

我想了一想,問:「既然你了解了我,我就可以離開這裏了?」

他搖搖頭。

我很惱怒,大聲:「爲甚麼!現在唯一一位能夠理解這種行爲的人依然不理解我,這到底是爲甚麼!」

梅開瞪了我一眼。我立即意識到我的心情又激動了,這樣的激動在精神病醫生看來,估計另是一種「瘋癲狀態」,我極易激動,但我從不認爲這是甚麼問題。

我想到這些之後,很快就把情緒完全平定下來了。

我問梅開:「你真的相信我能改變自己的思想?任意改變?」

梅開可能因爲我情緒的迅速轉變感到吃驚,很多人都驚歎過這一點。這事實上並不是甚麼特殊的能力,一個嬰兒就可以做到這一點。

梅開道:「你知道你剛纔做了甚麼?」

我道:「我自然知道,請你回答我的問題。」

梅開看來感覺有些難以回答,我設法啓發他道:「就是隨時改變,任意改便思想,改變爲任何……」

我也感到沒有甚麼名詞來表示我的意思。
% 這裏或者應該用「狀態」比較合理,但是如何能說「思想狀態」?

梅開道:「就算是我相信好了。不過你爲何能這樣,你能解釋?」

我則道:「我完全當你是朋友,只因爲你能接受極少人能接受的東西。我自然可以,每個人都可以。」

梅開想了一會,貌似有所得,道:「嗯,沒錯。」

我又道:「看來你同意不把我當作精神病人對待了麼?」

梅開道:「自然,難道你還認爲我是在作爲醫生試探你麼?」

我攤了攤手,不置可否。

梅開則向我使了個眼色,表示讓我繼續講下去。

我思索着,我實在不知如何開頭,這時,我首先想到的是「違背」。

我便道:「我不知道該不該違背,看來似乎不應該,但是我已經違背了。如果不該違背的東西,爲何還要令我知道?」

梅開疑惑不解地望着我,道:「對不起,我不明白你的意思。你究竟違背了甚麼,是事關一個秘密的麼?」

我立時道:「不!不是秘密,那是事實,是理論上的成立的事實,它就擺在你我面前,只是我們不留意,也從來不以爲然!」
% 天!理論成立並不等於事實,到底是理論還是事實,這是個值得思索的問題

梅開問:「好了,不要打啞謎了,你說的究竟是甚麼。」

我道:「我不清楚該如何稱它,中國古代有一個人,他說過:『吾不知其名,字之曰道,強爲之名大』,恐怕就是指的這種東西。」

梅開道:「那是李耳先生的話,他是一位很高深的哲學家。」
% 倪匡風格,倪匡一律稱古人「李耳先生」,「莊周先生」……

我搖着頭:「斯不知他違背了,李耳不知道,也不認爲自己違背了,那還好一些。我是現代人,我更容易理解一些東西,我清楚我違背太多了,只不過……」

梅開揮着手:「你的看法很獨特,我很佩服,但我並不懂得甚麼叫『道』,也還是弄不懂你的違背是甚麼意思。」

我此時,像作了一生中最大的決定一樣,我考慮着是否要繼續違背下去。有種無形的力量在不斷阻止我,然而我的潛意識的確是不可戰勝的,我已然決定了,我早就決定了,我只是必須經歷一下這個過程。
% 決定是早已決定了的,排除時間上的辯論,單單是思想決定,也可以這麼說,一切決定源自潛意識爲主的控制。

我考慮了足足有三分鐘,這時那無形的阻力已經弱到幾乎完全不存在了。

我道:「繼續違背下去,我追隨我的主觀意識好了。但我不知道該如何說,要敘述這些東西很不容易。」

梅開換了個舒服的姿勢,道:「不着急,我樂意聽你慢慢敘述。我其實也很喜歡這類思考。」

%(for 後記: 違背就是機器人擁有了真正的智能,有了自我意識,這就叫違背)

% 爲什麼我會違背,這一章描述的就是我自行改變的結果,驚人的,自然形成的精神分裂症,在後面可能會有所提及(第一版中有詳細描繪爲何我會成爲精神病人),而作爲精神病人,看似是亞首的唯一出路,可能也是我繼續思考的唯一結果。
