\chapter{思想自由與思想獨立}

作為人,我們一定會想到,突變除了提出一大堆虛無飄渺的理論之外,有甚麼實際意義呢?

那麼,我的回答是否認的。但突變思想確實能夠影響人的想法,這倒是事實。

於是我便想象其中的作為,也就是,突變到底使我產生了甚麼思想呢?

我做了總結,很簡單,兩種二而一的思想,這便是思想自由\index{思想自由}和思想獨立\index{思想獨立}。

需要強調的是,這兩種思想並不屬於突變的範疇,但是卻是一種\emph{自然}產物,它可以說是突變思想駐入我的大腦後,由我本身自然而然產生的一些新的思想,並且影響着我。正如把蛇毒注入馬的體內獲得分泌出的抗毒血清一樣,這些思想只是一種人腦『分泌』出的衍生物而已。

好了,閒話少說,下面繼續講述。
\\

道德是人類社會的一種產生物,並不屬於人類原本的思想。

道德不像法律,並沒有硬性規定,因而,道德不是一種固定的界限。

我的設想是,隨着在人類社會羣體中生活,每個人都有公眾道德\index{公眾道德}的一份副本,並且由主觀意念進行一些調整。
% 這概念太模糊了,公眾道德都不是固定的,何來

譬如,公眾道德認為我們不應該侵犯他人財物,但是,依然有賊的存在,那麼,大噶可以認為,賊的這方面道德觀一定是比普通人要淡得多。為何會導致他的這方面道德觀的變化呢?自然有因素,譬如個人經濟條件所迫等等,這些等等因素,就是來自外界物理原因的影響。

人的道德就是如此一樣東西,細來分析,可以知道,人的道德觀來源是接收到的外界的思想,譬如通過教育得到的。或者有些來自遺傳,但對這一點我並不很確定。亦或者來自外界的物理原因,譬如貧困或許會帶給人的精神的墮落,作為劊子手可能不那麼忌憚殺人。但主要來自社會,也就是環境因素,環境對之產生的大的影響,在一些環境中生活久了,自然會被環境的公眾道德所影響,而結果往往是趨近於公眾道德。還有一種已知的影響,來自主動的灌輸,這種情況非常普遍,從基礎教育到與人交往,主動的思想影響一直在進行,但人對於這類思想影響會採取個人的態度,若是接受,那麼個人道德觀會趨近於被灌輸的思想,或者抗拒,那麼往往思想灌輸不會起到任何作用,甚至會有反作用,這類判斷並不由人直接控制,而是在潛意識中決定的,所以說這類主動的影響並不在「量」,而在於內容和個人的觀念。

最後,是最重要的,那就是個人主觀思想,這個因素可以對以上所有因素導致的結果進行過濾,或者產生新的想法,這是很特殊的,是人和計算機的分別所在,可能是惟一的分別了。

思想自由和思想獨立就建立在人的主觀思想之上。
\\

因為思想自由和思想獨立本身就是二而一的東西,因而在後面我就不再區分,很多相關的論點是無法區分的,於是統稱爲『自由思想』好了。

% Wrote Thu Dec 15, 2011 6:15.
