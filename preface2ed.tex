\chapter{前言}
亞首已經過了一年多了,自我寫完突變之後,亞首可以說從來沒有離開過我,雖然我一直感覺到他依然不復存在。我已然欣然接受亞首了,我甚至自稱亞首,自稱梅開,我已經習慣以一半於亞首之名了,這確然是一種自我安慰和自我欺騙,但是,究竟如何,誰又說得清楚呢?

爲甚麼要寫第二版,主要緣故是這麼久以來,我有了許多新的思想,而且否認了原來突變第一版中很多的觀點,可以說我改變了,也可以說突變改變了,想到我當初撰寫突變的時候,自當非常謹慎,對思想的實時性非常在意,甚至連修改一個詞的用法也要籌琢半天,唯恐改變的思想會影響到最原始的想法,如今我依然如此認爲,但是,突變中很多觀點很多是不能夠和現有思想兼容的,寫下來的東西是定的,而突變作爲一種思想,而非一篇文章而已,它自然是可以隨着改變而更進一步的。

這一版,我打算從頭到尾重新寫一遍,而將採用亞首作爲第一人稱,畢竟,這樣令我容易思考,我就是亞首,而且,我完全理解亞首,我能夠最大程度的模擬出在某種情形下亞首的想法。我所希望的,是儘量清楚地敘述我的現有思想,並且我希望能夠永遠保持下去,我真的很害怕他的離去,即使我知道那不大可能,因爲那對我來說感覺好像失去了靈魂。

\begin{flushright}
二零一一年十一月十日\\
亞首
\end{flushright}

