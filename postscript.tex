\chapter{後記}
第二版中的敘述方式依然採用對話為主,其間,我沒有再強調很多顯而易見的東西,也沒有發表太多主觀的批評,這一年來,我在一些比較重大的問題上,有了很大突破,我自然採用更加完整的體系來敘述,而不像第一版一樣不得不讓梅開一開始就必須要接受幾種「突變公理」。而且,我在一些關鍵性問題的最後是留下空白的,沒有直接點出,這樣,以來可以令讀者自己稍微利用一點想象,不至於閱讀疲勞,另一個目的,可以讓讀者體驗一部分我當初有這類想法之後的那種激動,喜悅,成就感,雖然不至於那麼強烈,但是我想其中一部分發現可能依然能夠令你感到震撼。

可以說,第二版已經全然違背第一版的思想了,因為我在撰寫第二版時,完全沒有考慮如何與之兼容,二是二,一是一,一個新一些的思想,不應該受到原來的牽制,不是麼?

熟悉我的人,估計可以讀出這書中十之五六的典故出處,不熟悉我的人,有共同愛好者,也能看出一部分,但是沒有想象力是不行的。我對這篇東西的期望是很大的,而且做了大量投入,所以,我可以說,幾乎在每一句話裏,每一個動作,每一個看似不經意的細節都有典故可循,我將之選為突變的部分,是有意義的。

常見的一些典故,基於思想獨立,我將不再進行引用說明,既然我知道的,我就用了,否則突變就會成為一串出處字典,那是我不希望看到的。但為了尊重版權,我所借用的原文都是允許自由拷貝的,大段文字,或者被要求署名的小段文字,將會被加入參考資料,如果你發現我有任何侵權行為,請與我聯繫,我會儘快進行處理。

\flushright{亞首}
