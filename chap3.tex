\chapter{本質}

我道:「談到唯心主義\index{唯心主義},你想到了甚麼?」

梅開如實道:「我第一個想到的是靈魂\index{靈魂},第二個想到的是自我意識。」

我道:「那我便來向你分析一下靈魂。首先,你是否認爲生物\index{生物}是種很特殊的東西呢?」

梅開道:「生物簡直是太特殊了,生物是唯一有生命的東西。」

我則問之:「那麼你認爲甚麼叫生命\index{生命}呢?」

梅開道:「在生物學上來說,生命是指擁有繁衍能力的個體。這是最廣泛,最真切的解釋。所以,病毒被認爲是一種生物,即使它們只懂得繁殖。」

我道:「如果有一種生物,能夠幾乎完全不死,那它退化掉了繁衍能力,是否還算是生命呢?」

梅開托着頭想了想:「目前沒有發現任何這種生物,但我處於科學的角度看,並不能完全否認這類東西,但我不能確定它能不能算是生物。它或者有道理,說不定外星就有這類生物。」

我道:「你設想這類生物,我估且稱之爲生物,會是怎麼樣的呢?」

梅開道:「它若拋開繁衍,就沒有機會進化了,它一定是一種極度低等的生物,反之,如果它是經過進化而成爲的高智能生物,然後才退化掉了這個特性,它沒有了繁衍和死亡,卻可以有更高的智能。正如如果如果人類之間沒有死亡,知識就將可以大量堆積,普通人可以學習八十年,而他卻可以學習多得多的時間,他頭腦中的知識可以不斷增加,而不必像我們現在這樣需要用這麼低效的方式傳遞知識。」

我道:「不是這樣的,一個不死的生物並不是不能進化,它可以把自己身體的一部分器官拿掉,然後換上新的,如此往復,同樣可以挑選出最適合它的器官。那麼如此一來,它可以逐漸由低等變得高等了。」

梅開道:「這種情況倒也是可能的,好了,我承認這種東西爲生物了,因爲除了沒有繁衍,他們一樣可以成立,甚至比我們要多樣性的多。不過,我們老是在談論着一種完全不存在的生物幹甚麼?」

我道:「它們真的完全不存在麼?」

梅開想了一會,啊的叫了出來,他喊道:「不,它們絕不是生物,它們不是高等生物!」

待到他的心境平靜了,他才小心問道:「我真不願接受這個事實,但那確實是事實,他們的確是高等生物。」

我嘆:「可惜人類還千方百計想要去外星球尋找高智能生物。」

梅開道:「你證明了一種高智能生物的存在性,這消息向世界公佈出去,不知道有多轟動纔是。」
\\


我問道:「它們有靈魂麼?」

梅開道:「我不認爲它們有。」

我又問:「那麼人呢?病毒呢?」

梅開道:「根據G作家的觀點,靈魂就是人的記憶,存在的形式象一種電波,一直在這個人體外,在一個人死去之後,便變成遊離狀態,在某些時刻,可以影響另一個人腦部,形成類似『前世的記憶』的東西。」

我聽到這種觀點,道:「G作家的內容多半不可信,有些道理,是根本是無法成立的。」

梅開道:「這個觀點確然有些荒謬,譬如人的多種記憶,已經證明是在人腦的多種部位存儲了。如果人腦一些區域受損,那就會失憶。但是靈魂這種看似好像毫無道理的東西,卻又難以否認它存在,否則,如何解釋一個地球生命體的生命呢?如果講,靈魂離體,就失去生命,靈魂附體,就擁有生命,這是種最簡單的概括了,卻很能說明生命是甚麼東西。」

我則問道:「人死亡之後和人死亡之前的區別是甚麼?」

梅開道:「不能這麼說,目前醫學上判定人的死亡是根據腦死亡來判定的,但人的死亡並不是一瞬間的事,譬如心臟的死亡,肺的死亡,等等。從微觀來說可以說到一個細胞的死亡,如果這個細胞停止工作做了,它就是死亡了,而如果心臟中所有的細胞都死亡了,則可以稱這個心臟死亡了。」

我問:「你的意思是,人的死亡是分步驟的麼?腦死亡的時候,某些腦細胞逐漸停止工作,一直到整個腦的腦細胞都停止工作,這個步驟之間,死者是否有想法?」

梅開道:「理論上應該有的,但是事實上沒有哪個人能回答你這個問題,除非他經歷過。」

我又問:「那麼,死亡的原因是甚麼呢?」

梅開答道:「這可就數不盡了,不過從微觀來講,可以說,細胞死亡的唯一原因就是生存的條件得不到滿足,飢餓會使得它們得不到能量,寒冷會使他們達不到正常的工作溫度,這種情況,細胞就會死,從而導致一個人被餓死,凍死。」

我道:「可不可以這麼說,一個生物生存需要各種各樣的供應,這類供應可以滿足,它就存活,否則就死亡。」

梅開則道:「完全可以這麼說。」

我接着道:「你看看,這樣的話,靈魂的說法就顯得完全不必要了,科學完全可以解釋這種現象。」

梅開還有些不甘,我不等他發話,就道:「你看看,細胞和機器有甚麼區別?」說着,我順手把旁邊的電燈開關關掉了,房間的窗簾拉着,密不透光,所以這樣一來,房間暗了下來。我笑道:「我剛剛殺死了電燈。」

梅開昂着頭望着暗下去的燈。

我重新打開開關,笑道:「我順手之間又創造了一個靈魂,它使得這個細胞開始工作了。」

梅開跟着我笑起來。

梅開問我:「既然靈魂是如此,那麼生命到底是一種甚麼東西呢?我事實上已經接受你的說法了,但是我總不認爲那是一般平凡的東西。」

我想了一會,道:「我如此概括生命,生命體是物體的一種特殊結構。如此之外,我想不到如何用其它方法來描述生命了。」

梅開點點頭:「事實上,我的想法和你的差不多。」

梅開又笑起來:「確實是這樣,某宗教認爲,人是最高等的生物,是萬物之靈,生命都是寶貴的,他們卻不認爲植物算是生命,自然更不會像你一樣,認爲『它們』都算生命啦。」

我道:「自然,所謂『殺生』只不過是改變狀態而已,就如把冰拿到日光下暴曬,直到融化,這只是這種結構的狀態改變了。因此而有的罪惡感,實在是不該,某教鼻祖竟然還規定這些『改變狀態者』在改變狀態之後要到地獄接受懲罰,甚至還將轉換爲被改變狀態者的結構。」

梅開冷笑道:「洋人學中國人的笑話,你可曾聽過?洋人忘了如何說『請坐』,就說:『請把你的屁股放在椅子上……』」

我多少有點尷尬:「一點也不好笑,而且和我剛纔講的話,不發生任何關係。」

%(出處:G作家常用的一個笑話。)

梅開道:「事實上,你只要用通俗一點的說法,能更簡單地表達你的意思。你只消說『殺生有罪,死後要下地獄,殺牲畜者來世投胎爲牲畜』就可以了。」

我道:「好,我同意。類似的說法非常多。我認爲那樣的宗教應該毫無信徒纔對,然而看來它的信徒還不少。」
\\


我又道:「對於複製人,你現在看法如何?」

梅開道:「我想法自然和以前不同了,既然創造生命只是創造一種特定結構,上帝的工作就不再神聖,你可以做,我可以做,我們都一直在做。複製便複製了,無論是人,青蛙,薔薇,沒有甚麼分別,像是工廠裏生產鞋子一樣。」

我問道:「這樣你明白了,但是,人依然有非常重要的意義,我並不在於把人貶低。」

梅開道:「我想我懂你的意思,我們不斷創造生命,創造軀殼,殺死生命,等等都並無關係。人類之間對生命的重視,其間除了對同種物種之間的自然聯繫,還有另一個非常重要的因素。」

我問道:「你指甚麼?」

梅開道:「思想,人有思想,有記憶,這部分東西是無價的,對於人類來說,肉體是思想的唯一承載體,因而人纔會那麼看重的。人這麼長時間的進化以來,自然產生過非常多的知識,但直到幾千年前纔有人想到要把知識記錄下來。你也可以看到,在人類思想可以部分得到保留,而且只是極小的一部分,那麼在後天學習性的進步是多麼迅速。」

我想了一下,道:「我很贊同你的觀點。譬如,人如果能夠複製思想,可以把一個人的思想複製到複製人腦中,我看就不會有人那麼看重自己的生命了,或者,從另一個角度來說,他永生了。」

梅開笑道:「那正如把一塊已經備份過的硬盤拿去銷毀,自然不必再感到可惜了。」
\\


我問梅開:「事實上,我們是否真的可能複製思想?」

梅開道:「我作爲醫學方面的專家,倒是可以給你一個比較合理的答案。所謂複製思想,意思主要在於複製記憶,因爲思想是由記憶決定的。」

我又問:「那麼記憶是如何保存的呢?」

梅開道:「不像磁盤一樣,人的記憶存儲方式非常奇特。一個嬰兒生下來後,他的腦幾乎是空白的,此時,他只有一些本能的反應,在一些器官漸漸成熟之後,他會從外界得到信息,並且腦中的記憶區域開始存儲一些東西,腦存儲方式並不是綫性存儲,而是一種非常籠統的的,非常不可靠的抽象存儲法。」

我打斷他的話:「我不太明白你的『抽象存儲方式』是大概怎麼樣的。」

梅開向我解釋道:「人類目前還無法破譯人的存儲方式,所以我也無法具體地回答你。我們知道,人腦習慣將一些相關的東西關聯起來記憶,那樣比較容易記住,而且人的記憶並不是只有存在或者不存在兩種狀態,它可能是有『記憶深度』這樣一種量,但是科學研究發現,記憶深度並不是一個等速的量,也就是它的增加並不和某一簡單事件或環境相關,規律目前來說還完全不可循。再往下延伸開去,還是人類的未知領域。」

我又問:「照你看來,它會不會是類似結構方式存儲的。」

梅開道:「我的看法是,它不應該是結構式的存儲。記憶節點是相互關聯的的,而事實上,記憶節點的邊界甚至還很不明顯,我們實在無法描述它,那真是一種很奇妙的東西。」

我問道:「如果把人類腦部記憶區的神經細胞完全照原來的結合方式和狀態加到了另一個人腦中,另一個人是否能夠擁有他的思想,至少是部分的?」

梅開嘆了一口氣:「這個問題沒有人知道,然而我對之有一個猜想。我認爲人和人在成長過程中,腦的發展有不同的方式,細胞也有不同的結合方式,因而會產生各種性格的人,而且因爲記憶方式不同,有些人擅長記人,有些人則更擅長記文字。這就好像計算機的系統架構不同,二進制程序不一定具有可移植性,然而計算機架構是固定的,而人腦的結構卻每個人都不盡相同,除非在思想之間建立某種使之兼容的關係,然而這看來也是不太可能的。」

我道:「你講的不錯,如果要建立兼容關係,自然需要先對人腦有一定程度的瞭解,但是目前來說,人暫且是沒有辦法找到完全理解它的方式的。但是你的說法有誤,人思想之間存在一個兼容層。」

梅開稍微想了一下,又道:「是,但是這個兼容層並不能真實地傳遞信息,它不完善,很容易僞造,而且就人類天生的獸性而言,它簡直是一直在虛假下進行的。」

我笑了一下,道:「那既然是人類天生的,你還要稱之獸性,看來你的偏見還是很重啊!」
\\


梅開指着我的腦袋笑道:「看來你那裏那是一種非常平等的世界。」

我道:「你可以這麼說,在這個思想中,沒有任何的階級概念,甚至唯一可能產生方向的時間,都是不定向的。」

梅開道:「你的時間說又是怎麼一回事?」

我問梅開拿了一張紙和一支鉛筆,然後畫了一條綫。

梅開很是疑惑:「這是甚麼意思?」

我問:「你看來,這條綫是不是有方向?」

梅開點點頭,道:「是的,這條綫的方向就是你畫它的時間順序。」

我道:「把這條綫拿到計算機屏幕上,它依然是一條綫,但是經過放大之後,我們可以得知它是由一堆有規律的點構成的,這些點卻互不相干。」

梅開道:「你說的沒錯,這些點之間的規律就是它們互相貼近,而且連續,在宏觀看來,這就是一條綫。」

我又道:「這麼多點之間有順序之分麼?」

梅開道:「你的意思是,這些點象徵着每一個『時間節點』上的快照,它們之間的順序則是時間順序,對麼?」

我道:「正是此意,你可以看,它依然擁有順序的,但是當我們把它們拿到二維空間裏去觀察,它們就只是一大羣的點了,並沒有順序之分。」

梅開道:「把它放大,你可以看到每一個點上的快照,每一個快照都是那個時間點上的鏡像,它們之間本質是沒有聯繫的,但是它們呈現一定的規律,這就看來是有先後次序的了。」

我道:「對,看來你理解了。」

梅開沒有回答,他想了三分鐘,才面露驚色:「亞首,我發現你的類比中一個大的漏洞。計算機的屏幕或存儲是有精度限制的,而時間則不同,即使它看來如此,但是它依然是連續的,而不能分成一個個的『節點』。」

我拍着梅開的肩膀:「梅,你仔細想想,人類已經證明了的,它真的是連續的麼?」

梅開幾乎跳起來,他大口喘着氣,像是離開水的魚:「亞首!亞首!你,你……」
\\


我拍了拍梅開的肩膀,道:「來,梅,我來告訴你如何創造一個宇宙。」

梅開剛平靜下來,聽我這麼講,又瞪着我,像是看到了甚麼怪物一樣。

我道:「別這麼看着我,我說的雖然是理論,但是卻可行。」

我開始講述我的方法:「首先,需要確定的是宇宙是有基本屬性的,而且是有限個種類的基本屬性,雖然不一定是常數數量。」

梅開道:「我不能給你答覆,你能確定這一點麼?」

我道:「作爲天文學或微觀物理學的角度來看,這個問題沒有定論,而且可能永遠沒有定論。」

梅開道:「好傢伙,剛剛從生物學的未知領域回來,卻又涉及到物理學的未知領域了。」

我接着道:「但是在我的思想中,這個問題是有定論的,我目前無法向你解釋我的推論,因爲現在的對你來說,這套思想還有太多的未知數,但是,我是能夠證明它的。你現在需要的是暫時接受它,而在我完全向你描述完思想的時候,你自然會理解的。」

梅開道:「你確定你的思想能夠自圓其說?」

我道:「我不確定,事實上,我以前曾經認爲它是非常完善,堅不可摧的,然而結果是我幾乎改變了它的核心內容。目前我也如此認爲,但是我無法下定論,我想你能懂的。」

梅開道:「我看不出這有甚麼關係,那麼你只管講你的想法好了,我聽完之後再下結論。」

我開始敘述:「你有沒有聽說過『生命遊戲』?」

梅開道:「甚麼叫『生命遊戲』?」

我道:「那是一種計算機的程序,一般來說,這樣的遊戲擁有一塊矩形,以最小的格子爲基本單位,在一部分的格子上放上表示生命的東西。遊戲制定一些規則,它們的狀態只有兩種:生存,死亡,每一個格子中的生命的存活與否,在於周圍格子中存活生物的數量,一般來說,周圍過多或過少的存活的生命都會導致它死亡,比較適中的環境中,它可以存活,而且,在沒有存活生命的地方,如果環境生長,會在那裏繁衍一個一個生命出來。規則一般可以用計算機程序來模擬執行,程序總是迅速且準確的。」

%(按:這裏敘述很不清楚,相關描述可以參考維基百科的詞條)

梅開道:「那又如何?」

我道:「如果我說,這些簡單的規則之下會產生智能生物,你是否相信?」

梅開想了一會,道:「不大可能,它們只是數據,不是生物,它們甚至連形體都沒有。」

我又道:「不是這樣的,它們個體的規則非常簡單,但是在龐觀上,常常可以看得出很多複雜的結合,有如有生命一樣,它們會有聚合性,會有遷移性,會有類似繁衍的特性。」

梅開道:「無論怎麼說,這都是我們的程序賦予它的啊。」

我道:「不,我們只告訴它觀察身邊的生命,來斷定自己是否成活,我們完全沒有告訴它該如何聚在一起。那是由已有規律衍生出來的另一種規律,更複雜的。」

梅開道:「你可以這麼說,但是,這類程序和你所要講述的宇宙有什麼關係呢?」

我問道:「你認爲如果把生命遊戲複雜化,給它更高維度的空間模擬,給它更多一點的規律,譬如一個基本物體受到的影響和其它所有的物體都有關係,那它的複雜程度能達到多高呢?」

%(如果這樣的話,生命遊戲中的複雜程度一定是總的環境的大O,我們無法比宇宙複雜,不是麼?)

梅開聽到這話,揮着手,喃喃道:「不可能,不可能……」

每個人都看得出來,他並沒有否認這種情況的可能性,相反,他是認可了這種可能,但是卻實在不願接受。

我道:「如果這個宇宙存在有限的基本規則,那麼在理論上,它就可以被計算機程序模擬。」

梅開接着我的話:「而且……而且如果反過來說,如果它是被計算機模擬的,它就一定是存在有限的基本規則的。」

我又道:「而且計算機模擬的情況下,不可能有無窮大或者真正的無窮小,你要知道。」

梅開簡直是哭喪着臉:「可是現代科學已經證明有這些特性了……」

我問:「你無法證明這種情況不可能被模擬,是麼?」

梅開重重地叫道:「是!」

我又問:「你無法證明這不是模擬,是麼?」

梅開更加悲哀地道:「是!」

我攤了攤手:「這還可以看上去像遞歸調用。」

梅開更是「啊」了一聲,道:「還有!」

我安慰道:「其實事實並不一定是這樣的,這種情況只是可能。」

梅開道:「你不用安慰我,我能夠接受這個事實,我只是感到非常詫異。」

我很是不好意思地笑了笑,道:「是,這雖然不像數學那麼嚴謹,但是在那麼多事實的支持下,我更傾向於認爲它就是事實。」

我接着道:「如果不帶感情因素的話,不是很難接受。就像人類發現自己並不在宇宙中心一樣,如果被證明了,人就會接受這一點。」

梅開道:「一些具體的問題呢?譬如如何回溯?」

我道:「很難說,我估計很難回溯,但是根據複雜程度遞減的因素來看,我們說不定可以找到終點,但是起點,我們很難追溯到它們。」

梅開又道:「我們能否讓他們注意到我們呢?」

我道:「我想過這個問題,我認爲可以,完全可以,說不定現在他們就在注視着我們的談論,但是,由於一個智能生物體究竟該是甚麼形狀,甚麼狀態,怎樣一種結構,都是沒有定論的,因而他們無法再編寫一個程序專門在這個宇宙中搜尋智慧生物,而宇宙中的粒子實在太多了,他們不可能一一來分析,而且那樣也不見得有什麼結果。或者他們根本不在意是否有我們存在。」

梅開道:「亦或者,他們不破壞程序和人之間的聯繫界限。」

接着,梅開突然想起了一些東西,道:「天,時間不是單向的!流速也不是固定的!他們高興,可以把我們這裏停止一千年再開動,而我們渾然不知。」

我道:「對,這也是一種合理的解釋。」
\\


梅開道:「有些東西可能並不來自我們自己。」

我道:「試舉一例。」

梅開道:「僅僅說生物的遺傳信息系統吧。」

我聽到這裏,打斷他的話,驚叫起來:「天,你想到了,你早就想到了?」

梅開緩緩點頭:「在看到那些報告的時候,想象它的複雜程度和適配程度,你真得不得不驚歎於它。」

我道:「一定是那樣的,它是不自然的。它能夠可靠地存儲的數據,那存儲方式,那複製的方式,數據可以直接用來轉換並使用,都非常異特。」

梅開:「對,只可惜這只是一種猜想,完全無從證實,否則不知道會給人類羣體帶來多大的震撼。」

我道:「即使無法證實,那它也是成立的。正如當一個擁有人類語言分析的智能程序發現自己體內有0xdeedbeef這個常數時,我想他不可能不懷疑這東西是否程序自然產生的。」

%(出處:程序員有時慣用可以寫成十六進制的單詞deadbeef來做神奇數字)

我又向梅開解釋了一番這句話的意思。

我道:「我說的智能程序,就是這個意思。」

梅開表示不明白,我又解釋道:「我要說明的是,我們就是這樣的智能程序。」

梅開這才恍然大悟,笑道:「這是不錯的猜想。」

我道:「或者他們本來的目的就是觀察一組數據的變化的。我們就是因爲這個才被創造出來的。」

