\chapter{幻想世界}

梅開嘆道:「我們甚麼時候纔夠條件這樣去創造宇宙啊!」

我道:「我們已經在做了。」

梅開道:「我不懂你的意思。」

我道:「程序是虛擬的存在,意念也是虛擬的存在,不是麼?」

梅開道:「可以這麼說。」

我又道:「世界並不一定要如此完整,是麼?」

梅開道:「是。」

我道:「那麼,我們的意念難道沒有能力能力創建一個新的世界麼?」

梅開道:「你在說『夢』對麼?」

我道:「是,但不僅僅是夢,包括幻想,冥想,都有這個能力。」

梅開道:「嗯,你說得對。」

我道:「計算機和人腦,除了方式不同,其實本來就是一而二,二而一的東西。計算機模擬,和人腦模擬,在本質上沒有甚麼區別,人腦可能會更多的加入個人因素,加入一些想象的東西。」

梅開想了一下,道:「是的,人的夢境常常會和種種相關,而且並不像計算機一樣可以模擬微觀物體。」

我道:「我就擁有這樣的能力,我能夠製造幻想,並在其中。」

梅開作了一個驚訝的表情,道:「你的檔案裏提到你常常會在一些時候發呆。」

我道:「或者我幻想的時候看上去是那個樣子,事實上我不可能知道我是什麼樣的。」

梅開道:「你的幻想能力,你能進一步說明麼?」
\\


我道:「你以爲我從哪裏得到的這麼多奇怪的思想?」

梅開道:「難道是從幻想中得到的?真不可思議!」

這是顯而易見的,我只是道:「我想向你講述一些關於幻想和夢境的東西,以及我的相關能力。」

梅開道:「我很樂意聽這些。」

%/* Tuesday, 22th November 2011, 5:03 at the Morning */

我道:「首先熟悉一下我的一些名詞定義吧,我對一些相關的名詞有有些差異的定義。」

梅開道:「好。」

我道:「首先,對我來說,幻想指的是人在清醒時自行進入的一種冥想狀態,一般來說有可以模擬一種幻境,感受,圖像,等等都是模擬出來的,而且在幻想中依然很清醒。冥想一般來說專指不模擬幻境的幻想,在這個狀態下,可以進行思考問題,事實上,在冥想中思考問題簡直就是享受一般。再一個就是夢境,夢境是非自主進入的一種想象境界,一般是有圖像環境的,相對於幻想而言,夢境中人不是清醒的,而且不知道自己身在夢境,在這個狀態下,夢境顯得非常真實,一些不合理的地方在夢中會自動變得合理,相對冥想而言,夢境更容易涉及自己的潛意識,但一般來說夢境是不可控的,而且從夢中醒來,往往無法記住夢中的東西。」

梅開想了一會,像是要把這些東西灌到腦子裏一樣,然後點了點頭,表示他已經接受了。

我道:「你是否能回憶一下人進入夢境有什麼感覺?」

梅開道:「沒有人記得,當你快要睡着時,你的感覺就消失了,而且不會有任何記憶。」

我又問:「不,我指的是在有意識的時候,最後的一點感覺是什麼?」

梅開道:「我無法回答這個問題,我從來沒有試過留意這些,而且幾乎是留意不來的。」

我道:「我曾經花過幾個月在研究幻想和夢境上,你是否想聽聽我的想法和收穫?」

梅開道:「好。」

我道:「我最早開始留意這類事情是我發現我有一種特殊的能力,我在夢中知道自己的身份,會懷疑自己的所在,也就是說,我在夢中總是知道自己在做夢。」

梅開道:「這不很稀奇,很多人有過這種體驗。」

我道:「是,一開始我也沒有把這當回事過,我常常做同樣的夢,那是個噩夢。我明白這是個夢,但是那種夢中帶來的恐懼讓我非常難受,於是我便試圖醒來。」

梅開道:「天!你能做到?」

我道:「我做到了,但是這事情並不那麼簡單,我設法醒來,沒有實際的方法,並不是說在夢中自殺就可以達到目的的,我只是有一個意念,堅定那個意念,那就是『我在夢中,我要醒來』,而在進行一段時間的非常困難的對自己注入這個意念時,我會發現我醒來了,躺在牀上。」

梅開道:「這倒是很有趣的事情。」

我道:「不,隨即我就又發現自己並沒有醒來,我依然在夢中,我使勁移動我的手,睜開我的眼睛,但是我無法控制現實中的人做任何動作,我感受得到現實中躺在牀上的那個傢伙,但我只能控制我在夢中的身份。我發現自己還在夢中之後,我會繼續用那個方法,而結果總能醒來,但並沒有真正的醒來。那感覺真的很難說,有點像一個程序的運行指針在跳出一個個調用堆棧,然而好像堆棧是無窮層的一樣。」

梅開道:「我想象得到那種感覺,那麼結果是如何呢?」

我道:「在跳出不知道多少層調用棧後,我自然還是醒來了,醒來後,我會發現我不太想移動了,但是很輕鬆,因爲我已經可以隨時自由擡起手,睜開眼了。然後我就開始回憶夢境。」

梅開做了個手勢,讓我接着講下去。

我道:「此時,夢境往往很容易回憶,我可以記起很多東西,甚至一些細節。」

梅開道:「這倒是個很大的發現。」

我道:「在這之後一段時間。我就考慮另一個情況,如果我在夢中知道其身後,並不着急醒來,而是嘗試『遊覽』夢境,又會如何。」

梅開道:「你做得到麼?」

我道:「我做不到,主要的因素是我在夢中無法接受現實中來的想法,我只是知道自己在做夢,我像是根本把現實世界忘記了一樣,我根本不會想到有曾經有過這個想法。也就是說,我無法把現實中的思想傳遞給夢境中使用。」

梅開道:「那真可惜,但看上去你一定還有別的發現。」

我道:「不錯,但是我很快就可以在夢中使用現實中的思想,甚至完全用現實中的思想來控制夢境中的的角色的思想了。」

梅開有些驚愕:「你是怎麼做到的。」
\\


我道:「我首先要向你講述一下進入夢境的一個個層次,才能敘述這個方法。」

根據長時間的觀察和經驗,我把進入夢經分爲五個層次。

首先是第一層,我名之爲:「物理催眠」,在這時候,一般來說我會在音樂中進行簡單的自我催眠,擁有自我意識,感知的到四肢的存在,但在催眠下,我會漸漸失去對肢體的控制。

然後進入了第二層,一般而言,在任何時候都可以通過幻想或冥想進入這一層次,我名之曰:「可控冥想狀態」,此時對一切的感官都已經非常淡化了,但是在需要的時候還可以想起來,並加以控制。思想是自由的,在對外界感官淡化之後,思想會變得異常清晰,常常可以進行很好的思考。

隨即會進入另一層,我名之曰:「可知冥想狀態」,這個狀態就意味着將要入睡了,這時候,思想會常常不受控制地胡思亂想,而我們並不知道會想到什麼,我認爲此時應該思想是能夠很容易訪問到潛意識。

但是我們如果關注一下,還是可以知道自己在想什麼,當然,在有這個念頭的時候,就會暫時的回到「可控冥想狀態」,而這兩個狀態轉換非常模糊。但是要知道,在可知冥想狀態下是可以興起一些念頭並回到可控狀態的,如果完全沒有了這些念頭,那麼就進入了第四層。

第四層我名之:「混沌狀態」,這個狀態中沒有人知道發生了什麼,不再注意思想了,至於是否還會有思想活動,我無法回答這個問題,即使有也絕不會知道。沒有感覺,但是這個狀態下人非常舒服,這個狀態下可能會做夢,但是即使有也是無法知道的,更不用提記得住了。

第五層就是我們常常說的:「夢境」,這是個非常奇怪的一個狀態,這時候有特殊的腦部運轉,有思想,有記憶,有邏輯,有感官。夢境是純思想的世界。但是在夢境中發生的大部分故事我們是可以回憶的。

我向梅開講了這些之後,又花了不少工夫使他明白其中的意義。
\\


我道:「我實現能夠傳入思想所運作的就在可控冥想層,這個層本來就和幻想差不多,所以很容易控制,而我要做的,就是儘量在可知冥想狀態和可控之間切換,在可控的時候儘量思考我希望傳入夢境的思想,而在可知冥想層儘量回到可控狀態。往往,就可以在這些切換之間把可控冥想層中的思想深化到潛意識中。」

梅開道:「這是可能的。那麼在混沌狀態下你難道還能保證那些『數據』不受到干擾麼?」

我道:「不能,事實上不會有多少干擾,在控製得好的情況下,甚至可能跳過混沌狀態直接來到夢境,那就是最爽的體驗了。」

梅開道:「那豈不是你可以完整地保留思想?」

我道:「自然,那才叫一個自在,現實的思想比夢境中角色的思想要聰明的多,所以,在一個有想象力的思想進入夢境中,簡直就是有超能力一樣。」

梅開不解地問:「譬如?」

我道:「譬如我會考察夢中環境的真實性,如果我看到一張寫着字的紙,在夢中的人知道它大概寫了什麼就夠了,而我不僅同樣知道那寫着什麼,我還可以質疑它,觀察它是否真的寫着那些東西。」

梅開又問:「那麼結果如何?」

我笑道:「夢境中的一切是用來騙夢中的不會質疑的小人的,而對於現實中的思想者,簡直是漏洞百出。就拿那張紙來說,我不止在一次夢境中不止一次試圖質疑它,結果當我放大來看時,它是雜亂無章的字,甚至大小不一,完全和我知道它的意思不同,很明顯是夢境爲了適應這種想法臨時模擬了一些東西,然而自然是不合理之至的。」

梅開道:「那麼有趣!你還玩了些什麼花樣?」

我道:「夢境是我模擬出來的,所以我的意念可以命令夢境模擬者把模擬的工作交給夢中的我。」

梅開道:「哦!天哪!」

我接着道:「不錯,我就可以隨意創建我希望的東西了,譬如那張紙,我可以讓它上面寫任何文字,只要我想就可以了。」
\\


夢境控制是我掌握的最有趣的技能了,我確然在夢中能夠擁有自己,那是另一個虛幻的世界,在那個世界中,我是規則制定者,那麼意味着我可以實現一切,只要我想。到那個時候,你才意味得到什麼叫「思想無限」,想象力萬歲!

我夢中除了能按常理控制之外,我還控制別的東西,譬如,我最擅長的就是在夢境中懸浮一個物體,下面的空氣好像可以托着它一樣,使它飄浮在空中。

我還可以以極快的速度飛行,以任何方式移動,我很喜歡夢中的飛行,而在我速度很快或者飛得太高時,我甚至還會有一點恐懼,感覺完全是真的一樣。

我嘗試過用意念控制計算機開關,控制光標移動,等等。那自然是在夢境中做的。

我還往往更換環境,譬如若我不喜歡這樣的世界,我可以把環境更換到美國的一個大城市,我可以任意修改其中的一磚一瓦,當然,這些的模板是我意念中的映像,我甚至去過外星球,參關過那裏的世界。

我可以洞察人心,這一點是在現實中完全無法做到的。在夢境中,每個人都像透明的一樣,他們絕不會有私心,包括我,永遠不必在夢中防範有人向你進攻。這不算什麼,我還可以控制夢中人的想法。

%/** for following will asked from Mee 有一點我需要說明的是,在夢中,有無限的自由,但是絕對不會生起權力的想法 **/

